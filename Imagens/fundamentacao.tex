\chapter{Fundamentacao Teorica}
\section{HISTORIA DO FUNDADOR}
\hspace*{0.5cm}Robert Stephenson Smyth Baden-Powell nasceu no dia 22 de fevereiro de 1857 em Londres, capital da Inglaterra. Sua mãe era Henrietta Smyth e seu pai era o Reverendo Baden-Powell, professor de Oxford que tinha conhecimento nas áreas de matemática, astronomia, filosofia e teologia. B-P perdeu seu pai quando tinha pouco mais de 2 anos de idade e ficou com sua mãe e mais 6 irmãos.
\hspace*{0.5cm}Robert Smyth viveu uma vida ao ar livre e realizando acampamentos com seus irmãos, onde já demonstrava habilidades e vocação para a vivência nesse ambiente. Na escola B-P não era o aluno mais brilhante, mas se destacava pelo seu grande interesse e curiosidade, o que o levava a participar de muitas atividades oferecidas na escola. Aos 19 anos de idade colou grau na Escola Charterhouse e logo em seguida surgiu a oportunidade do mesmo ir para a Índia como subtenente do Regimento que formara a ala direita da Cavalaria na Célebre.
\hspace*{0.5cm}Com todo o conhecimento e por sempre se destacar na vivência ao ar livre, já era de se esperar a brilhante carreira que ele teria no serviço militar. Aos 26 anos chegou a patente de capitão e rapidamente foi conseguindo outras promoções. Em 1899 já tinha sido promovido a Coronel.
\hspace*{0.5cm}No coração da África do Sul, havia uma cidade muito cobiçada por ser vista como um ponto estratégico comercial, chamada Mafeking. Com o rompimento da relação entre Inglaterra e o governo da República de Transval, Baden-powell recebeu ordens para organizar um batalhão de carabineiros montados para proteger Mafeking. A guerra iniciou-se em 13 de outubro de 1899 e B-P defendeu aquela terra durante 217 dias apesar das tropas inimigas serem três vezes maior do que o seu exército. Finalmente em 18 de maio a tropa de socorro conseguiu abrir caminho para auxiliá-lo e resultou no sucesso em sua missão. Através desse feito, conseguiu a patente de Major-General e ali já havia se tornado um herói, reconhecido e aclamado por todos.
\hspace*{0.5cm}Em 1899, Robert Stephenson escreveu o livro: Aids To Scouting, um guia para recrutas do exército. Ao retornar da África em 1901, notou e ficou surpreso por sua popularidade, visto que, seu livro estava sendo muito utilizado em escolas masculinas. Posto isso, enxergou o desafio de ajudar jovens no crescimento pessoal, já que o livro escrito especialmente para adultos possuía uma grande força motivacional. Por conseguinte, pensou em escrever um livro voltado para os jovens.
\hspace*{0.5cm}Após juntar e ler vários livros, buscando métodos utilizados durante todas as épocas para educar e adestrar jovens, cuidadosamente B-P foi desenvolvendo a ideia do Escotismo.
Para comprovar a eficácia da ideia e de todos os métodos buscado, desenvolvido e adaptado por ele mesmo, no verão de 1907, reunido com 20 rapazes, montou um acampamento na ilha de Brownsea, no Canal da Mancha. E com grande êxito, este vinha a ser o primeiro acampamento escoteiro.
\hspace*{0.5cm}Em 1908 foi lançado o seu livro de adestramento “Escotismo para rapazes”, que logo em seguida, foi o impulsionador para o surgimento de patrulhas e tropas escoteiras por vários países do mundo, e dessa forma, iniciava-se o movimento escoteiro.
\hspace*{0.5cm}O movimento cresceu tanto que Baden-Powell em 1910 decidiu seguir uma nova jornada em sua vida, onde pediu demissão do exército para poder se dedicar à vida ao ar livre, treinando jovens para que viessem a ser bons cidadãos.\\
\hspace*{0.5cm}A partir de 1912, B-P começou a visitar vários países levando o movimento escoteiro, primeiro passo esse para tornar o escotismo uma fraternidade mundial. O movimento continuou crescendo, até que no início da primeira guerra mundial houve a necessidade de dar uma pausa nas atividades. Porém, aos poucos, novamente Baden-Powell voltava a suas atividades normais dentro do escotismo, e em 1920, escoteiros de todo o mundo se reuniram em Londres para a primeira concentração internacional de escoteiro, denominada primeiro Jamboree Mundial.
\hspace*{0.5cm}Quando o movimento completou 21 anos, já havia mais de dois milhões de jovens fazendo parte, e nessa ocasião, Robert recebeu do Rei Jorge V, o título de Lord Baden-Powell of Gilwell.
B-P aos 80 anos de idade retornou para a África, onde passou os últimos anos de sua vida em Kenia ao lado da sua esposa, Lady Baden-Powell.