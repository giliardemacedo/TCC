\begin{center}
\Large{\textbf{RESUMO}}
\end{center}
O escotismo � um movimento educacional para os jovens com o aux�lio de adultos. Esse movimento trata-se de uma grande fraternidade com mais de 40 milh�es de membros em 216 pa�ses. Desde de 1907, quando Baden-Powell fundou o escotismo, a quantidade de membros e associados aos grupos vem aumentando com uma grande quantidade de jovens e adultos espalhados por diversos grupos no mundo inteiro. O movimento escoteiro possui uma estrutura �nica, onde possui uma hierarquia a ser seguida e dessa forma, manter a base s�lida criada pelo fundador, para que assim, possa se conduzir com maior qualidade essa grande massa de associados.Hoje temos muitos livros, cartilhas e manuais que s�o utilizados desde o pr�prio jovem, com intuito de aprender e aprimorar t�cnicas educacionais para o crescimento pessoal, ou at� mesmo livros utilizados pelos adultos que visam o desenvolvimento pessoal como chefe, buscando cada vez mais conhecimento para melhor conduzir os jovens ao seu redor. Como � citado, o escotismo � rico em uma vasta biblioteca com livros e materiais para todos os gostos, mas para quem prefere o uso tecnol�gico a Uni�o dos Escoteiros do Brasil que � a associa��o nacional, disp�e de um sistema chamado Paxto e dois aplicativos que s�o o mAPPa Jovem e mAPPa Adulto. Essas ferramentas s�o utilizadas por todos os grupos e possuem funcionalidades gen�ricas em que os grupos as utilizam para acompanhamento de progress�o dos jovens, disponibilizar um acervo de materiais, realizar frequ�ncia dos jovens nas atividades, entre outras funcionalidades. Apesar de serem programas de uma excelente qualidade, s�o programas com uma base de funcionalidades gen�ricas. Chefes da Unidade Escoteira Local (UEL) de Tenente Laurentino, buscam melhorias para as atividades e acompanhamento dos escoteiros nos dias atuais, tendo em vista a pequena quantidade de adultos dispon�veis para atividades no dia a dia. Atrav�s, principalmente dessa necessidade, foi pensado em desenvolver um sistema capaz de saciar as necessidades internas do grupo, e nesse sistema espec�fico para a UEL, foi pensado em ferramentas para agilizar e acelerar a elabora��o de atividades de forma autom�tica, assim como, a implementa��o de um sistema de caixa que seja f�cil de ser manipulado pelo usu�rio, entre outras funcionalidades de grande import�ncia para os jovens e adultos do grupo.

\vspace{5cm}
\textbf{Palavras-chaves:}\hspace*{0.2cm}Browsea, scout management, activity management, box control.

%\newpage
